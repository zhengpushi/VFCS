\documentclass{article} % PDFTex, XeLaTex 都支持。

% 中文环境
%\documentclass[hyperref, UTF8]{ctexart} %若选择{ctexart}则直接支持中文,下面的{ctex}要去掉。
\usepackage[UTF8]{ctex} %中文配置。
%\usepackage[UTF8, heading=false, scheme=plain]{ctex} %加入scheme=plain,行距、中西文公共字符都有变化。

% 版面设置
\usepackage{geometry}
\geometry{a4paper}

% 额外的功能
\usepackage{authblk} %添加机构,需要安装preprint包
\usepackage{amsthm} %证明环境
\usepackage{amsmath} %数学公式
\numberwithin{equation}{section} % 公式按章节编号
\usepackage{amssymb}
\usepackage{multirow} % multirow
\usepackage{booktabs} % toprule, midrule, bottomrule

% 表格的单元格内换行、对齐功能
\usepackage{makecell}

% 图片支持
\usepackage{graphicx} %添加图片
\graphicspath{{figures/}}
\usepackage{float} % 控制图片是否浮动:[htbp]浮动,或[H]禁止浮动

% PDF索引
\usepackage{hyperref}

% 首行缩进
\usepackage{indentfirst} % 首行缩进支持
\setlength{\parindent}{2em} % 首行缩进两个汉字

% 列表样式定制
\usepackage{enumerate}
\usepackage{enumitem}
\setlist[enumerate,1]{label=(\arabic*).,font=\textup,leftmargin=14mm,labelsep=1.5mm,topsep=0mm,itemsep=-0.8mm}
\setlist[enumerate,2]{label=(\alph*).,font=\textup,leftmargin=14mm,labelsep=1.5mm,topsep=-0.8mm,itemsep=-0.8mm}
%\setlist{nosep} % 取消行间空行

% 自定义命令
\newcommand{\SL}{\rule{.3em}{.3pt}} % 定义短下划线

% 名称、作者、机构
\title{Simplification for Arithmetic Expressions}
\author{Zhengpu Shi}
\affil{V1.0}
%\affil{南京航空航天大学}
%\date{2020年12月29日} %注释后显示为编译时日期


\begin{document}

% 生成标题
\maketitle
%\newpage
% ++++++++++++++++++++++++++++++++++++++++++++++++++++++++++++++++++++++++++++++++++++++++++++++++++

% 生成目录
%\tableofcontents
%\newpage
% ++++++++++++++++++++++++++++++++++++++++++++++++++++++++++++++++++++++++++++++++++++++++++++++++++

%\listoffigures
%\newpage
% 生成图片列表,请删除上面两行注释

%\begin{figure}[H]%[htbp]
%\centering
%\includegraphics[scale=0.6]{gradient.jpg}
%\caption{this is a figure demo}
%\label{fig:label}
%\end{figure}

% \newpage
% ++++++++++++++++++++++++++++++++++++++++++++++++++++++++++++++++++++++++++++++++++++++++++++++++++

\subsection{Introduction}
我的研究方向是可靠飞行控制软件,目前工作是使用COQ来做飞控中数学推导与控制模型推导的正确性验证。
在飞行器姿态表示这一子系统中用到了四元数数学理论。

由于COQ系统没有四元数理论,我们给出了四元数的一个实现,并对四元数的乘法公式给出了一份证明。
我将含有i,j,k复数形式的四元数乘法按照i,j,k的公理性质展开为实数域或整数域上的算术运算,并最终完成了四元数乘法公式的证明。
为实现乘法公式推导,需要处理 (a + b i + c j + d k) * (e + f i + g j + h k) 这样的表达式。
我们按照 i * i = -1, i * j = k 等定义,最终处理为 s + t i + u j + v k 这种形式。
我们构造了每一步推导过程。并最终在满足 ring 性质的通用数据结构下完成了证明。
然后利用Module、Functor产生了 Z、Real 上的四元数算法和性质。


\end{document}