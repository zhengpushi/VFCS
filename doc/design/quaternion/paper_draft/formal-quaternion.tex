% !TEX TS-program = pdflatex
% This is samplepaper.tex, a sample chapter demonstrating the
% LLNCS macro package for Springer Computer Science proceedings;
% Version 2.21 of 2022/01/12
%
\documentclass[runningheads]{llncs}
%

\usepackage[T1]{fontenc}
% T1 fonts will be used to generate the final print and online PDFs,
% so please use T1 fonts in your manuscript whenever possible.
% Other font encodings may result in incorrect characters.
%

\usepackage{graphicx}
% Used for displaying a sample figure. If possible, figure files should
% be included in EPS format.

\usepackage{amssymb}
% mathematical symbols
%

\usepackage{amsmath}
% support bmatrix

\usepackage{enumerate}
% Support more enumerate style

\usepackage{pifont}
% more style for bullets
%

\usepackage{makecell}
% cell style, support wrap
%

\usepackage[
bookmarksnumbered=true,
bookmarksopen=false,
bookmarksopenlevel=1,
colorlinks=true,  % it must be enabled to enable other color option
citecolor=black,
filecolor=black,
linkcolor=black,
urlcolor=black,
pdfstartview=Fit
]{hyperref}
% set bookmark
%
% If you use the hyperref package, please uncomment the following two lines
% to display URLs in blue roman font according to Springer's eBook style:

\usepackage{minted}
%% Format and highlight programming language source code with minted
% 风格    关键字    类型    注释     备注
% default 绿色      红色    深青斜体
% colorful绿色      深蓝    灰色斜体
% abap    浅蓝      浅蓝    灰色斜体
% autumn  深蓝      青色    灰色斜体
% sas     浅蓝      浅蓝    绿色斜体
% vs      浅蓝      深青    绿色
% igor    浅蓝      浅蓝    红色斜体
\usemintedstyle{autumn}%default}%autumn}
% delimit with # (Jing hao))
\newminted[coqJ]{coq}{fontsize=\footnotesize,%\small,
  mathescape, texcomments, escapeinside=\#\#,
  linenos,numbersep=-5pt}
% delimit with ? (Wen hao)
\newminted[coqW]{coq}{fontsize=\footnotesize,
  mathescape, texcomments, escapeinside=??,
  linenos,numbersep=-5pt}

\usepackage{makecell}
%% Support wrap in cell of table

\usepackage{color}
\renewcommand\UrlFont{\color{blue}\rmfamily}
%
\usepackage{marvosym}
%\usepackage{ifsym}
% corresponding author
%

\usepackage{xparse}
% Support advanced command definition

%% Short name for special nouns.
% (1) prover, languages, project, etc.
\newcommand{\Coq}{{\sc Coq}}
\newcommand{\HOLLight}{{\sc HOL-Light}}
\newcommand{\IsaHOL}{{\sc Isabelle/HOL}}
\newcommand{\Lean}{{\sc Lean}}
\newcommand{\Ocaml}{{\sc Ocaml}}
\newcommand{\MC}{{\sc Mathematical Components}}
\newcommand{\Coquelicot}{{\sc Coquelicot}}
\newcommand{\CoqPackIdx}{{\sc Coq Package Index}}
\newcommand{\CoqStdLib}{{\sc Coq Standard Library}}
\newcommand{\CoqMatrix}{{\sc CoqMatrix}}
\NewDocumentCommand{\Cmat}{g}{\IfNoValueTF{#1}{\mathtt{Cmat}}{\mathtt{Cmat}(#1)}}
\NewDocumentCommand{\Cvec}{g}{\IfNoValueTF{#1}{\mathtt{Cvec}}{\mathtt{Cvec}(#1)}}
\newcommand{\Clen}[1]{\Vert #1\Vert}

% (2) math symbols
\newcommand{\bN}{\mathbb{N}}
\newcommand{\bR}{\mathbb{R}} 
\newcommand{\bQ}{\mathbb{Q}}
\newcommand{\bZ}{\mathbb{Z}}
\newcommand{\bC}{\mathbb{C}}

% (3) abbreviations for math equation writing
\newcommand{\bmatL}{\begin{bmatrix}}
\newcommand{\bmatR}{\end{bmatrix}}

\begin{document}
%
\title{Formal Quaternions in Coq
%\thanks{Supported by organization x.}
}
%
%\titlerunning{Abbreviated paper title}
% If the paper title is too long for the running head, you can set
% an abbreviated paper title here
%
\author{ZhengPu Shi
  % \orcidID{0000-0001-5151-507X}
  % \and Gang Chen$^{(\textrm{\Letter})}$\orcidID{0000-0003-1790-2484}
}
%
%\authorrunning{F. Author et al.}
% First names are abbreviated in the running head.
% If there are more than two authors, 'et al.' is used.
%
%\institute{Princeton University, Princeton NJ 08544, USA \and
%Springer Heidelberg, Tiergartenstr. 17, 69121 Heidelberg, Germany
%\email{lncs@springer.com}\\
%\url{http://www.springer.com/gp/computer-science/lncs} \and
%ABC Institute, Rupert-Karls-University Heidelberg, Heidelberg, Germany\\
%\email{\{abc,lncs\}@uni-heidelberg.de}}
% \institute{Nanjing University of Aeronautics and Astronautics, NanJing 211106, China
% \email{zhengpushi@nuaa.edu.cn}~~~\email{gangchensh@qq.com}}

%
\maketitle              % typeset the header of the contribution
%
\begin{abstract}
  (preserved)
  \keywords{Coq Proof Assistant
    \and Quaternion
% \and formal matrix theory
% \and interface and implementation
% \and isomorphic mapping
    .
}
\end{abstract}
%
%
%
% ##################################################################################################

% Here are demo code for figure, table, reference etc.
% As shown in Fig.~\Ref{fig_title1}.
% \begin{figure}[htbp]
% {\centering
% \includegraphics[width=0.7\textwidth]{figures/fig1.pdf}
% \caption{Title of the figure}
% \label{fig_title1}
% }\end{figure}

% As shown in Table~\ref{tab_title1}.
% \begin{table}\begin{center}
%   \caption{Title of the table.}
%   \label{tab_title1}
%   \begin{tabular}{|r|c|c|}
%     \hline
%     No & Title & Remark \\
%     \hline
%     1 & ab & cd
%     \hline
%   \end{tabular}
% \end{center}\end{table}

% \begin{coqJ}
%   Definition a : R := 1.
% \end{coqJ}


\section{Introduction}\label{intro}

We will study the complex, quaternion, rotation, etc. in a formal style in Coq.
These concepts are parts of basis for control system.

The quaternion is related to 3D rotation, while the complex number is related to 2D rotation.
The quaternion has many properties similiar with complex number, thus, study complex number will be helpful for quaternion.

\section{2D rotation and complex number}

\subsection{Definition of complex number}

Given a complex number $z\in\bC$, it could be represented as $z=a+bi$, here, $a,b\in\bR, i^2=-1$.
$a$ is called the \textit{Real Part} of $z$, $b$ is called the \textit{Imaginary Part} of $z$.
\begin{equation}
  \mathtt{Re}(z)\triangleq a,\qquad \mathtt{Im}(z)\triangleq b
\end{equation}

We can consider that $z=a+bi$ is a Linear Combination on the Basic of $\{1,i\}$.
So, the Coordinate of it could be used to represent $z$, which is a 2D Vector.
We define the conversion function from a complex number to a 2D vector as follows.
\begin{equation}
  \Cvec{z}\triangleq\bmatL a\\b\bmatR 
\end{equation}

A 2D vector is corresponding to a Point in Complex Plane, also corresponding to a Point in 2D Plane.

In conclusion, a complex number is corresponding to a 2D vector, and corresponding to a point in 2D plane.
Therefore, when studying the motion in 2D plane, we have geometric forms and corresponding analytic forms.

\subsection{Operations and properties of complex number}

\subsubsection{Addition and subtraction}
Given two complex numbers $z_1=a+bi, z_2=c+di$, then the addition and subtraction of them are defined as follows.
\begin{equation}
  \mathtt{Cplus}(z_1,z_2)\triangleq (a+c)+(b+d)i,\quad
  \mathtt{Csub}(z_1,z_2)\triangleq (a-c)+(b-d)i,
\end{equation}
which are denoted as $z_1+z_2, z_1-z_2$ seperately.
  

\subsubsection{Multiplication}
We will give two style definitions for multiplication of complex numbers.
\begin{enumerate}[(1)]
\item \textbf{ALGEBRAIC STYLE}
  Given two complex numbers $z_1=a+bi, z_2=c+di$, then the multiplication of them are defined as follows.
  \begin{equation}\label{complex_mult_alg}
    \mathtt{Cmult_A}(z_1,z_2)\triangleq (a+bi)(c+di)=(ac-bd)+(bc+ad)i,
  \end{equation}
  which is denoted as $z_1 *_A z_2$.
  We call the \eqref{complex_mult_alg} is a algebraic style definition.

  We can rewrite this result with the multiplication of a matrix and a vector.
  \begin{equation}
    z_1z_2=(ac-bd)+(bc+ad)i=\bmatL a&-b\\b&a\bmatR \bmatL c\\d\bmatR 
  \end{equation}
  Here, $\bmatL c\\d\bmatR =\Cvec{z_2}$ is the vector form of $z_2$, and $\bmatL a&-b\\b&a\bmatR $ is a matrix determined by $z_1$.
  We find that any complex number has a $2\times 2$ matrix form, and this is a one-to-one correspondence.
  We define the matrix form of a complex number $z$ as follows.
  \begin{equation}
    \Cmat{z}\triangleq \bmatL a&-b\\b&a\bmatR 
  \end{equation}
  We can find that the multiplication of two complex numbers $z_1$ and $z_2$ is equivalent to the multiplication of two matrices $\Cmat{z_1}$ and $\Cvec{z_2}$ (note that vector is column matrix).
\item \textbf{MATRIX STYLE}
  Now, we can use matrix form to define the multiplication of two complex numbers.
  \begin{equation}\label{complex_mult_matrix}
    \mathtt{Cmult_M}(z_1,z_2)\triangleq\bmatL a&-b\\b&a\bmatR \bmatL c&-d\\d&c\bmatR 
    =\bmatL ac-bd&\quad -(bc+ad)\\bc+ad&\quad ac-bd\bmatR ,
  \end{equation}
  which is denoted as $z_1 *_M z_2$.
  We call the \eqref{complex_mult_matrix} is a matrix style definition.
  
  Also, We can observed that the righthand of \eqref{complex_mult_matrix} is exactly equal to $\Cmat{\mathtt{Cmult_A}(z_1,z_2)}$, which is the matrix form of multiplication of $z_1$ and $z_2$.
  This demonstrating that these two definitons are equivalence.
\end{enumerate}

Because we know that multiplication of complex number is commutative, we can verify this property use the two definitions \eqref{complex_mult_alg} and \eqref{complex_mult_matrix} seperately.
The correctness of them are proved by the derivations of \eqref{cmult_comm_proof_alg} and \eqref{cmult_comm_proof_matrix}.

\begin{equation}\label{cmult_comm_proof_alg}
  \begin{aligned}
    \mathtt{Cmult_A}(z_1,z_2)
    &=(a+bi)(c+di)\\
    &=(ac-bd)+(bc+ad)i\\
    &=(ca-db)+(cb+da)i\\
    &=(c+di)(a+bi)\\
    &=\mathtt{Cmult_A}(z_2,z_1)
  \end{aligned}
\end{equation}

\begin{equation}\label{cmult_comm_proof_matrix}
  \begin{aligned}
    \mathtt{Cmult_M}(z_1,z_2)
    &=\bmatL a&-b\\b&a\bmatR \bmatL c&-d\\d&c\bmatR  \\
    &=\bmatL ac-bd&\quad -(bc+ad)\\bc+ad&\quad ac-bd\bmatR \\
    &=\bmatL ca-db&\quad -(cb+da)\\cb+da&\quad ca-db\bmatR \\
    &=\bmatL c&-d\\d&c\bmatR \bmatL a&-b\\b&a\bmatR  \\
    &=\mathtt{Cmult_M}(z_2,z_1)
  \end{aligned}
\end{equation}

Because $\Cmat$ is an bijective function, we claim that $\mathtt{Cmult_A}$ is isomorphic to $\mathtt{mmul}$ over $\Cmat$,
here $\mathtt{mmul}$ is matrix multiplication.
The homomorphism relation between them is:
\begin{equation}\label{cmula_cmulm_hom}
  \Cmat(z_1 *_A z_2)=\Cmat(z_1) *_M \Cmat(z_2)
\end{equation}

Similarly, the addition and subtraction of complex numbers is isomorphic to the addition and subtraction of matrices, which will be omitted here.

Now, we can check the matrix form of some specify complex numbers.
\begin{equation}
  \Cmat{1}=\bmatL 1&0\\0&1\bmatR =I,\quad
  \Cmat{i}=\bmatL 0&-1\\1&0\bmatR 
\end{equation}

Because $i^2=-1$ and the homomorphism relation in \eqref{cmula_cmulm_hom}, there must be $\Cmat{i^2}=\Cmat{-1}$.
It can be proved here.

\begin{equation}
  \Cmat{i^2}=\Cmat{i *_A i}
  =\bmatL 0&-1\\1&0\bmatR \bmatL 0&-1\\1&0\bmatR 
  =\bmatL -1&0\\0&-1\bmatR 
  =-I
  =\Cmat{-1}
\end{equation}

Afterwards, we will write $z_1z_2$ to donate these two multiplication definitions.

\subsubsection{Conjugate of complex number}
Given a complex number $z=a+bi$, the conjugate of it is defined as follows.
\begin{equation}
  \bar{z}=a-bi
\end{equation}

\subsubsection{Magnitude of complex number}
Given a complex number $z=a+bi$, the magnitude of it is defined as follows.
\begin{equation}
  \Clen{z}=\sqrt{a^2+b^2}
\end{equation}

We claim that $\Clen{z}=\sqrt{z\bar{z}}$, which is proved as follows.
\begin{equation}
  \Clen{z}^2=a^2+b^2=(a+bi)(a-bi)=z\bar{z}
\end{equation}
    
\subsection{Complex number multiplication and 2D rotation}

We can analysis the function $\Cmat$, and find the relation between multiplication and rotation.

Given a complex $z=a+bi$, then:
\begin{equation}
  \Cmat{z}=\bmatL a&-b\\b&a\bmatR 
  =\sqrt{a^2+b^2}
  \bmatL \frac{a}{\sqrt{a^2+b^2}}&\frac{-b}{\sqrt{a^2+b^2}}\\
    \frac{b}{\sqrt{a^2+b^2}}&\frac{a}{\sqrt{a^2+b^2}}\bmatR 
  =\Clen{z}
  \bmatL \frac{a}{\Clen{z}}&\frac{-b}{\Clen{z}}\\
    \frac{b}{\Clen{z}}&\frac{a}{\Clen{z}}\bmatR 
\end{equation}

According the defnition of trigonometric functions, we can get these definitions.
\begin{equation}
  \theta=\mathtt{atan2}(b,a),\quad
  \cos(\theta)=\frac{a}{\Clen{z}},\quad
  \sin(\theta)=\frac{b}{\Clen{z}}
\end{equation}

Thus, we can get
\begin{equation}
  \begin{aligned}
    \Cmat{z}&=\bmatL a&-b\\b&a\bmatR \\
            &=\Clen{z}\bmatL \cos(\theta)&-\sin(\theta)\\\sin(\theta)&\cos(\theta)\bmatR \\
            &=\bmatL \Clen{z}&0\\0&\Clen{z}\bmatR 
              \bmatL \cos(\theta)&-\sin(\theta)\\\sin(\theta)&\cos(\theta)\bmatR 
  \end{aligned}
\end{equation}

Here, we have splitted the original matrix to the multiplication of two matrices.
The left one $\bmatL \Clen{z}&0\\0&\Clen{z}\bmatR $ is scaling matrix,
and the right one $\bmatL\cos(\theta)&-\sin(\theta)\\\sin(\theta)&\cos(\theta)\bmatR $ is rotation matrix.

Let's check the transformation to two basis vectors $v_1=\bmatL 1\\0\bmatR $ and $v_2=\bmatL 0\\1\bmatR $ over these matrices.

\begin{enumerate}[(1)]
\item
  \begin{equation}
    \bmatL a&-b\\b&a\bmatR \bmatL 1\\0\bmatR 
    =\bmatL \Clen{z}&0\\0&\Clen{z}\bmatR \bmatL \cos(\theta)&-\sin(\theta)\\\sin(\theta)&\cos(\theta)\bmatR \bmatL 1\\0\bmatR
    =\bmatL \Clen{z}&0\\0&\Clen{z}\bmatR \bmatL \cos(\theta)\\\sin(\theta)\bmatR\\
  \end{equation}
  It shows that, the $v_1$ is first rotated anticlockwise by $\theta$ degree, and then scaled $\Clen{z}$ times.
\item
  \begin{equation}
    \bmatL a&-b\\b&a\bmatR \bmatL 0\\1\bmatR 
    =\bmatL \Clen{z}&0\\0&\Clen{z}\bmatR \bmatL \cos(\theta)&-\sin(\theta)\\\sin(\theta)&\cos(\theta)\bmatR \bmatL 0\\1\bmatR
    =\bmatL \Clen{z}&0\\0&\Clen{z}\bmatR \bmatL -\sin(\theta)\\\cos(\theta)\bmatR\\
  \end{equation}
  It shows that, the $v_2$ is first rotated anticlockwise by $\theta$ degree, and then scaled $\Clen{z}$ times.
\end{enumerate}

Therefore, left multiply the complex number $z_1=a+bi$ to a complex number $z_2$, is corresponding to two effects, which first rotate $z_2$ anticlockwise by $\theta=\mathtt{atan2}(b,a)$ degree, then scale it by $\Vert z_1\Vert=\sqrt{a^2+b^2}$ times.

If the magnitude of the complex number is $1$, then the geometric meaning of complex number multiplication is just a pure rotation.
So, if we want to purly rotate a vector $v$ in 2D plane by $\theta$ degree, we could do it by a matrix multiplication.
\begin{equation}\label{pure_rot_2D_mmul}
  v'= \bmatL \cos(\theta)&-\sin(\theta)\\\sin(\theta)&\cos(\theta)\bmatR v
\end{equation}

We can peform pure rotation by complex multiplication too.
\begin{equation}\label{pure_rot_2D_cmul}
  v'= (\cos(\theta)+i\sin(\theta))v
\end{equation}

\subsection{Polar form of complex number}

By Euler's Formula,
\begin{equation}
  \cos(\theta)+i\sin(\theta)=e^{i\theta}
\end{equation}
a complex number $z$ could be written as
\begin{equation}
  z=\Clen{z}\bmatL \cos(\theta)&-\sin(\theta)\\\sin(\theta)&\cos(\theta)\bmatR
  =\Clen{z}(\cos(\theta)+i\sin(\theta))=\Clen{z}e^{i\theta}
\end{equation}

We get the polar form of complex number, after denoted the magnitude with a symbol $r$.
\begin{equation}
  z=re^{i\theta}, where~r=\Clen{z},\theta=\mathtt{atan2}(b,a)
\end{equation}

Now, we can give the formula of transformation a vector $v$ by the vector $re^{i\theta}$.
\begin{equation}
  v'=re^{i\theta}v
\end{equation}

If the length $r$ of the rotation vector is $1$, then the pure rotation formula is as follow.
\begin{equation}\label{pure_rot_2D_polar}
  v'=e^{i\theta}v
\end{equation}

Note that, three pure rotation formulas of 2D plane in \eqref{pure_rot_2D_mmul}, \eqref{pure_rot_2D_cmul}, and \eqref{pure_rot_2D_polar} are equivalent, and we can use them as needed.

\subsection{Composition of rotation in 2D}
Given a vector $v=x+yi$, and two unit complex numbers $z_1=\cos(\theta)+i\sin(\theta), z_2=\cos(\phi)+i\sin(\phi)$,
then the composite of $v$ first rotating $z_1$ and then rotating $z_2$ is equivalent to $z_{net}$.
\begin{equation}
  v'=z_1v,\quad v''=z_2(z_1v)=(z_2z_1)v=z_{net}v,\quad z_{net}\triangleq z_2z_1
\end{equation}

Because the multiplication of complex number is commutative, thus:
\begin{equation}
  z_{net}= z_2z_1 = z_1z_2
\end{equation}

Additionally, we can give a direct expression for $z_{net}$.
\begin{equation}
  z_{net}= \cos(\theta+\phi)+i\sin(\theta+\phi)
\end{equation}

Because,
\begin{equation}
  \begin{aligned}
    z_{net}
    &= (\cos\theta+i\sin\theta)(\cos\phi+i\sin\phi)\\
    &= (\cos\theta\cos\phi-\sin\theta\sin\phi)+i(\cos\theta\sin\phi+\sin\theta\cos\phi)\\
    &= \cos(\theta+\phi) + i\sin(\theta+\phi)
  \end{aligned}
\end{equation}
Therefore, we can conclude that the composition of two rotation in 2D is also a rotation, the result is independent to the order of each rotation operation, and the the equivalent rotation angle is the sum of their rotation angles.

\section{3D rotation}

There are several ways to represent rotation in 3D.
\begin{enumerate}[i.]
\item The representation by Euler Angle is related to the sequence of three axes and have Gimbal Lock problem.
\item The representation by Axis-angle is a more general case of rotations.
\item The representation by quaternion is another way which have better mathematical properties.
\end{enumerate}

We will use right-hand coordinate here to specify the rotation direction, that is: if we use thumb finger to point the positive direction of rotation axis $u$, then bending direction of the other four fingers is the positive direction of rotation.

In the representation by Axis-angle, we need four variables to define a rotation, including the 3D coordinate $x,y,z$ of the rotation axis $u$, and the rotation angle $\theta$.
Such that we have four Degree of Freedom (DOF).

We can observed that the Axis-angle method has one more DOF then Euler Angle method, which has only three DOF.
We have obtained the magnitude of the rotation axis $u$, after given the three coordinates of $u$.
However, we only need the rotation direction when considering a rotation, and need not to give the magnitude.
Defining a direction in 3D space requires only two variables (E.g., the angle between any two coordinate axes).

To eliminate the redundant DOF, i.e. the magnitude of rotation axis, we can specify the magnitude as an arbitrary constant.
In practice, we usually specify the magnitude to be 1.
This is a convention for defining directions in mathematics and physics, and it usually facilitates subsequent calculations.

We can convert an arbitrary 3D vector to unit vector by doing normalization.
\begin{equation}
  \hat{u}=\frac{u}{\Clen{u}}
\end{equation}




% ##################################################################################################
\section{Conclusion}\label{conclusion}
  (preserved)

% ##################################################################################################
\subsection*{Acknowledgments}
% I would like to thank xxx for their research on xx formalization techniques, and my colleagues for their discussions and suggestions.
  (preserved)

%
% ---- Bibliography ----
%
% BibTeX users should specify bibliography style 'splncs04'.
% References will then be sorted and formatted in the correct style.
%
% \bibliographystyle{splncs04}
% \bibliography{mybibliography}
%
\begin{thebibliography}{99}

\bibitem{ref_overview_formal_methods}
Wang J, Zhan NJ, Feng XY, Liu ZM. Overview of formal methods. Ruan Jian Xue Bao/Journal of Software, 2019, 30(1):33-61(in Chinese with English abstract). \doi{doi:10.13328/j.cnki.jos.005652}

\bibitem{ref_fem}
Chen G, Shi ZP. Formalized engineering mathematics. Communications of the CCF, 2017, 13(10)(in Chinese with English abstract).

\bibitem{ref_coq}
Coq Development Team. The Coq Reference Manual 8.13.2.

\end{thebibliography}

\end{document}
