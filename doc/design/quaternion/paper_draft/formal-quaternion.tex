% !TEX TS-program = pdflatex
% This is samplepaper.tex, a sample chapter demonstrating the
% LLNCS macro package for Springer Computer Science proceedings;
% Version 2.21 of 2022/01/12
%
\documentclass[runningheads]{llncs}
%

\usepackage[T1]{fontenc}
% T1 fonts will be used to generate the final print and online PDFs,
% so please use T1 fonts in your manuscript whenever possible.
% Other font encodings may result in incorrect characters.
%

\usepackage{graphicx}
% Used for displaying a sample figure. If possible, figure files should
% be included in EPS format.

\usepackage{amssymb}
% mathematical symbols
%

\usepackage{pifont}
% more style for bullets
%

\usepackage{makecell}
% cell style, support wrap
%

\usepackage[
bookmarksnumbered=true,
bookmarksopen=false,
bookmarksopenlevel=1,
colorlinks=true,  % it must be enabled to enable other color option
citecolor=black,
filecolor=black,
linkcolor=black,
urlcolor=black,
pdfstartview=Fit
]{hyperref}
% set bookmark
%
% If you use the hyperref package, please uncomment the following two lines
% to display URLs in blue roman font according to Springer's eBook style:

\usepackage{minted}
%% Format and highlight programming language source code with minted
% 风格    关键字    类型    注释     备注
% default 绿色      红色    深青斜体
% colorful绿色      深蓝    灰色斜体
% abap    浅蓝      浅蓝    灰色斜体
% autumn  深蓝      青色    灰色斜体
% sas     浅蓝      浅蓝    绿色斜体
% vs      浅蓝      深青    绿色
% igor    浅蓝      浅蓝    红色斜体
\usemintedstyle{autumn}%default}%autumn}
% delimit with # (Jing hao))
\newminted[coqJ]{coq}{fontsize=\footnotesize,%\small,
  mathescape, texcomments, escapeinside=\#\#,
  linenos,numbersep=-5pt}
% delimit with ? (Wen hao)
\newminted[coqW]{coq}{fontsize=\footnotesize,
  mathescape, texcomments, escapeinside=??,
  linenos,numbersep=-5pt}

\usepackage{makecell}
%% Support wrap in cell of table

\usepackage{color}
\renewcommand\UrlFont{\color{blue}\rmfamily}
%
\usepackage{marvosym}
%\usepackage{ifsym}
% corresponding author
%

%% Short name for special nouns.
% (1) prover, languages, project, etc.
\newcommand{\Coq}{{\sc Coq}}
\newcommand{\HOLLight}{{\sc HOL-Light}}
\newcommand{\IsaHOL}{{\sc Isabelle/HOL}}
\newcommand{\Lean}{{\sc Lean}}
\newcommand{\Ocaml}{{\sc Ocaml}}
\newcommand{\MC}{{\sc Mathematical Components}}
\newcommand{\Coquelicot}{{\sc Coquelicot}}
\newcommand{\CoqPackIdx}{{\sc Coq Package Index}}
\newcommand{\CoqStdLib}{{\sc Coq Standard Library}}
\newcommand{\CoqMatrix}{{\sc CoqMatrix}}
% (2) math symbols
\newcommand{\bN}{\mathbb{N}}
\newcommand{\bR}{\mathbb{R}}
\newcommand{\bQ}{\mathbb{Q}}
\newcommand{\bZ}{\mathbb{Z}}
\newcommand{\bC}{\mathbb{C}}

\begin{document}
%
\title{Formal Quaternions in Coq
%\thanks{Supported by organization x.}
}
%
%\titlerunning{Abbreviated paper title}
% If the paper title is too long for the running head, you can set
% an abbreviated paper title here
%
\author{ZhengPu Shi
  % \orcidID{0000-0001-5151-507X}
  % \and Gang Chen$^{(\textrm{\Letter})}$\orcidID{0000-0003-1790-2484}
}
%
%\authorrunning{F. Author et al.}
% First names are abbreviated in the running head.
% If there are more than two authors, 'et al.' is used.
%
%\institute{Princeton University, Princeton NJ 08544, USA \and
%Springer Heidelberg, Tiergartenstr. 17, 69121 Heidelberg, Germany
%\email{lncs@springer.com}\\
%\url{http://www.springer.com/gp/computer-science/lncs} \and
%ABC Institute, Rupert-Karls-University Heidelberg, Heidelberg, Germany\\
%\email{\{abc,lncs\}@uni-heidelberg.de}}
% \institute{Nanjing University of Aeronautics and Astronautics, NanJing 211106, China
% \email{zhengpushi@nuaa.edu.cn}~~~\email{gangchensh@qq.com}}

%
\maketitle              % typeset the header of the contribution
%
\begin{abstract}
  (preserved)
  \keywords{Coq Proof Assistant
    \and Quaternion
% \and formal matrix theory
% \and interface and implementation
% \and isomorphic mapping
    .
}
\end{abstract}
%
%
%
% ##################################################################################################

% Here are demo code for figure, table, reference etc.
% As shown in Fig.~\Ref{fig_title1}.
% \begin{figure}[htbp]
% {\centering
% \includegraphics[width=0.7\textwidth]{figures/fig1.pdf}
% \caption{Title of the figure}
% \label{fig_title1}
% }\end{figure}

% As shown in Table~\ref{tab_title1}.
% \begin{table}\begin{center}
%   \caption{Title of the table.}
%   \label{tab_title1}
%   \begin{tabular}{|r|c|c|}
%     \hline
%     No & Title & Remark \\
%     \hline
%     1 & ab & cd
%     \hline
%   \end{tabular}
% \end{center}\end{table}


\section{Introduction}\label{intro}
  (preserved)

\begin{coqJ}
  Definition a : R := 1.
\end{coqJ}
  

% ##################################################################################################
\section{Conclusion}\label{conclusion}
  (preserved)

% ##################################################################################################
\subsection*{Acknowledgments}
% I would like to thank xxx for their research on xx formalization techniques, and my colleagues for their discussions and suggestions.
  (preserved)

%
% ---- Bibliography ----
%
% BibTeX users should specify bibliography style 'splncs04'.
% References will then be sorted and formatted in the correct style.
%
% \bibliographystyle{splncs04}
% \bibliography{mybibliography}
%
\begin{thebibliography}{99}

\bibitem{ref_overview_formal_methods}
Wang J, Zhan NJ, Feng XY, Liu ZM. Overview of formal methods. Ruan Jian Xue Bao/Journal of Software, 2019, 30(1):33-61(in Chinese with English abstract). \doi{doi:10.13328/j.cnki.jos.005652}

\bibitem{ref_fem}
Chen G, Shi ZP. Formalized engineering mathematics. Communications of the CCF, 2017, 13(10)(in Chinese with English abstract).

\bibitem{ref_coq}
Coq Development Team. The Coq Reference Manual 8.13.2.

\end{thebibliography}

\end{document}
